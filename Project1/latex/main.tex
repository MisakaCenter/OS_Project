% !TEX program = pdflatex
% !TEX options = -synctex=1 -interaction=nonstopmode -file-line-error "%DOC%"
% 作业模板
\documentclass[UTF8,10pt,a4paper]{article}
\usepackage{ctex}
\newfontfamily\menlo{MONACO.TTF}
\usepackage{amsmath}
\usepackage{diagbox}
\usepackage{float}
\usepackage{listings}
\usepackage{multirow}
\usepackage{tabularx}

\usepackage{url}
\usepackage{xcolor}
\newcommand{\tabincell}[2]{\begin{tabular}{@{}#1@{}}#2\end{tabular}}

\lstset{
    breaklines,                                 % 自动将长的代码行换行排版
    extendedchars=false,                        % 解决代码跨页时,章节标题,页眉等汉字不显示的问题
    backgroundcolor=\color[rgb]{0.96,0.96,0.96},% 背景颜色
    keywordstyle=\color{blue}\bfseries,         % 关键字颜色
    identifierstyle=\color{black},              % 普通标识符颜色
    commentstyle=\color[rgb]{0,0.6,0},          % 注释颜色
    stringstyle=\color[rgb]{0.58,0,0.82},       % 字符串颜色
    showstringspaces=false,                     % 不显示字符串内的空格
    numbers=left,                               % 显示行号
    numberstyle=\tiny\menlo,                    % 设置数字字体
    basicstyle=\small\menlo,                    % 设置基本字体
    captionpos=t,                               % title在上方(在bottom即为b)
    frame=single,                               % 设置代码框形式
    rulecolor=\color[rgb]{0.8,0.8,0.8},         % 设置代码框颜色
}  

\usepackage{pythonhighlight}
\usepackage{listings}
\usepackage{xcolor}
\usepackage{graphicx}
\usepackage[a4paper,left=2cm,right=2cm,top=2cm,bottom=2cm]{geometry}
\usepackage{fancyhdr}
% \catcode`\。=\active
% \newcommand{。}{.}
\newcommand{\CourseName}{操作系统(Operating System)}
\newcommand{\CourseCode}{CS307 \& CS356}
\newcommand{\Semester}{2020-2021学年第二学期}
\newcommand{\ProjectName} {Project 1: Introduction to Linux Kernel Modules} 
\newcommand{\StudentName}{刘涵之}
\newcommand{\StudentID}{519021910102}
\usepackage[vmargin=1in,hmargin=.5in]{geometry}
\usepackage{fancyhdr}
\usepackage{lastpage}
\usepackage{calc}
\pagestyle{fancy}
\fancyhf{}
\fancyhead[L]{\CourseName}
\fancyhead[C]{\ProjectName}
\fancyhead[R]{\StudentName}
\fancyfoot[R]{\thepage\ / \pageref{LastPage}}
\setlength\headheight{12pt}
\fancypagestyle{FirstPageStyle}{
    \fancyhf{}
    \fancyhead[L]{\CourseName\\
        \CourseCode\\
        \Semester}
    \fancyhead[C]{\large\bfseries\ProjectName \\}
    \fancyhead[R]{Name: \makebox[\widthof{\StudentID}][s]{\StudentName}\\
        ID : \StudentID\\
        }
    \fancyfoot[R]{\thepage\ / \pageref{LastPage}}
    \setlength\headheight{36pt}
}
\usepackage{amsmath,amssymb,amsthm,bm}
\allowdisplaybreaks[4]
\newtheoremstyle{Problem}
{}
{}
{}
{}
{\bfseries}
{.}
{ }
{第\thmnumber{ #2}\thmname{ #1}\thmnote{ (#3)} 得分: \underline{\qquad\qquad}}
\theoremstyle{Problem}
\newtheorem{prob}{题}
\newtheoremstyle{Solution}
{}
{}
{}
{}
{\bfseries}
{:}
{ }
{\thmname{#1}}
\makeatletter
\def\@endtheorem{\qed\endtrivlist\@endpefalse}
\makeatother
\theoremstyle{Solution}
\newtheorem*{sol}{解}
% \usepackage{graphicx}
\begin{document}
\thispagestyle{FirstPageStyle}

\section{Prepare Experiment Environment}
In this section, I use VMware WorkStation to install the Ubuntu operating system. Then, I compile and upgrade the linux kernel to the latest version (5.11.3).
\subsection{VMware WorkStation and Ubuntu OS}
I install the \textbf{VMware Workstation 16 Pro} and download the image of Ubuntu OS (\textbf{ubuntu-20.04.2.0-desktop-amd64.iso}). I create a new virtual machine and choose this image to install.

\begin{figure}[H]
\begin{minipage}[H]{0.5\linewidth}
    \centering
    \includegraphics[width=160pt]{1.png}
    \caption{VMware WorkStation}
    \label{1}
\end{minipage}
\begin{minipage}[H]{0.5\linewidth}
    \centering
    \includegraphics[width=230pt]{2.png}
    \caption{Install Ubuntu OS}
    \label{2}
\end{minipage}
\end{figure}

Then I check the version of linux kernel, using the following instruction in bash.

\begin{lstlisting}
uname -a
\end{lstlisting}

\begin{figure}[H]
    \centering
    \includegraphics[width=380pt]{3.png}
    \caption{Check the Kernel Version}
    \label{3}
\end{figure}

In Figure \ref{3}, it shows that the current kernel version is \textbf{5.8.0}.

\subsection{Change the Package Source}
To download packages more faster, i change the source of the system package manager to SJTUG source. I follow the document in \underline{https://mirror.sjtu.edu.cn/docs/ubuntu} and using the following instructions.

\begin{lstlisting}
sudo gedit /etc/apt/sources.list
\end{lstlisting}

\begin{lstlisting}
deb https://mirrors.sjtug.sjtu.edu.cn/ubuntu focal main restricted
deb https://mirrors.sjtug.sjtu.edu.cn/ubuntu focal-updates main restricted
deb https://mirrors.sjtug.sjtu.edu.cn/ubuntu focal universe
deb https://mirrors.sjtug.sjtu.edu.cn/ubuntu focal-updates universe
deb https://mirrors.sjtug.sjtu.edu.cn/ubuntu focal multiverse
deb https://mirrors.sjtug.sjtu.edu.cn/ubuntu focal-updates multiverse
deb https://mirrors.sjtug.sjtu.edu.cn/ubuntu focal-backports main restricted universe multiverse
deb http://archive.canonical.com/ubuntu focal partner
deb https://mirrors.sjtug.sjtu.edu.cn/ubuntu focal-security main restricted universe multiverse
\end{lstlisting}

\begin{lstlisting}
sudo apt update
sudo apt upgrade
\end{lstlisting}

To prepare for compiling, i need to install some packages essential for building.

\begin{lstlisting}
sudo apt-get install kernel-package git fakeroot build-essential
sudo apt-get install ncurses-dev xz-utils libssl-dev bc flex bison
\end{lstlisting}


\subsection{Compile and Upgrade Linux Kernel (Recommended)}
To understand how kernel is installed, I decide to upgrade my kernel.

I find the latest release of linux kernel on \underline{www.kernel.org}

\begin{figure}[H]
    \centering
    \includegraphics[width=300pt]{4.png}
    \caption{the Latest Kernel Version}
    \label{4}
\end{figure}

I download the \textbf{linux-5.11.3.tar.xz} and then move it to \textbf{/usr/src}

\begin{lstlisting}
cp linux-5.11.3.tar.xz /usr/src
\end{lstlisting}

Then i unzip the file using the following instructions.

\begin{lstlisting}
sudo xz -d linux-5.11.3.tar.xz
sudo tar xvf linux-5.11.3.tar
\end{lstlisting}

\begin{figure}[H]
    \centering
    \includegraphics[width=380pt]{5.png}
    \caption{File in /usr/src}
    \label{5}
\end{figure}

Then i use the following instruction to set the configurations, the menu is displayed as Figure \ref{6}. I use the default configurations.

\begin{lstlisting}
sudo make menuconfig
\end{lstlisting}

\begin{figure}[H]
    \centering
    \includegraphics[width=380pt]{7.png}
    \caption{sudo make menuconfig}
    \label{7}
\end{figure}

\begin{figure}[H]
    \centering
    \includegraphics[width=380pt]{6.png}
    \caption{Menu of configurations}
    \label{6}
\end{figure}

Finally, i run the following instruction to compile the kernel with 8 threads compile in parallel.

\begin{lstlisting}
sudo make -j8
\end{lstlisting}

After 30 minutes, the compilation is complete.

\begin{figure}[H]
    \centering
    \includegraphics[width=280pt]{9.png}
    \caption{sudo make -j8}
    \label{9}
\end{figure}

I run the following instruction to install modules.

\begin{lstlisting}
sudo make modules_install
\end{lstlisting}

\begin{figure}[H]
    \centering
    \includegraphics[width=280pt]{10.png}
    \caption{sudo make modules\_install}
    \label{10}
\end{figure}

Then i can install the new kernel!

\begin{lstlisting}
sudo make install
\end{lstlisting}

\begin{figure}[H]
    \centering
    \includegraphics[width=400pt]{11.png}
    \caption{Install the new kernel}
    \label{11}
\end{figure}

Then i reboot the system and check the kernel version again. It shows that the new kernel (5.11.3) is installed successfully.

\begin{figure}[H]
    \centering
    \includegraphics[width=420pt]{12.png}
    \caption{Check the new kernel}
    \label{12}
\end{figure}

% \begin{prob}[问题标题]
%     此乃问题描述。
% \end{prob}
% \begin{sol}
%     此为解。
% \end{sol}

\section{Kernel Modules Overview}
In this section i will talk about the programming project at the end of Chapter 2\cite{book1}.
\subsection{Task 1 - Simple Module}

\begin{lstlisting}
sudo make
sudo dmesg -C
sudo insmod simple.ko
dmesg
sudo rmmod simple
dmesg
\end{lstlisting}

\begin{lstlisting}[language = c]
#include <linux/init.h>
#include <linux/module.h>
#include <linux/kernel.h>
#include <linux/hash.h>
#include <linux/gcd.h>

int simple_init(void)
{
       printk(KERN_INFO "Loading Module\n");
       printk(KERN_INFO "GOLDEN_RATIO_PRIME: %llu\n", GOLDEN_RATIO_PRIME);
       return 0;
}

void simple_exit(void) {
	printk(KERN_INFO "GCD(3300, 24) = %lu\n", gcd(3300,24));
	printk(KERN_INFO "Removing Module\n");
}

module_init( simple_init );
module_exit( simple_exit );

MODULE_LICENSE("GPL");
MODULE_DESCRIPTION("Simple Module");
MODULE_AUTHOR("MisakaCenter");
\end{lstlisting}

\begin{lstlisting}
obj-m := simple.o
all:
	make -C /usr/src/linux-5.11.3/ M=$(shell pwd) modules
clean:
	make -C /usr/src/linux-5.11.3/ M=$(shell pwd) clean
\end{lstlisting}

\begin{figure}[H]
    \centering
    \includegraphics[width=380pt]{simple1.png}
    \caption{simple.ko}
    \label{88}
\end{figure}

\begin{lstlisting}[language = c]
printk(KERN_INFO "(Loading) Jiffies: %lu\n", jiffies);
printk(KERN_INFO "HZ: %d\n", HZ);
\end{lstlisting}

\begin{lstlisting}[language = c]
printk(KERN_INFO "(Removing) Jiffies: %lu\n", jiffies);
\end{lstlisting}

\begin{figure}[H]
    \centering
    \includegraphics[width=380pt]{simple2.png}
    \caption{Print jiffies and HZ}
    \label{7}
\end{figure}

\subsection{Task 2 - Jiffies Module}



\begin{lstlisting}[language = c]
#include <linux/init.h>
#include <linux/module.h>
#include <linux/kernel.h>
#include <linux/proc_fs.h>
#include <asm/uaccess.h>
#include <linux/jiffies.h>

#define BUFFER_SIZE 128

#define PROC_NAME "jiffies"

ssize_t proc_read(struct file *file, char *buf, size_t count, loff_t *pos);

static struct proc_ops proc_ops = {
        .proc_read = proc_read
};

int proc_init(void)
{
        proc_create(PROC_NAME, 0, NULL, &proc_ops);
        printk(KERN_INFO "/proc/%s created\n", PROC_NAME);
	return 0;
}

void proc_exit(void) {
        remove_proc_entry(PROC_NAME, NULL);
        printk( KERN_INFO "/proc/%s removed\n", PROC_NAME);
}

ssize_t proc_read(struct file *file, char __user *usr_buf, size_t count, loff_t *pos)
{
        int rv = 0;
        char buffer[BUFFER_SIZE];
        static int completed = 0;
        if (completed) {
                completed = 0;
                return 0;
        }
        completed = 1;
        rv = sprintf(buffer, "jiffies: %lu\n", jiffies);
        copy_to_user(usr_buf, buffer, rv);
        return rv;
}

module_init( proc_init );
module_exit( proc_exit );

MODULE_LICENSE("GPL");
MODULE_DESCRIPTION("Jiffies Module");
MODULE_AUTHOR("MisakaCenter");
\end{lstlisting}

\begin{lstlisting}
obj-m := jiffies.o
all:
	make -C /usr/src/linux-5.11.3/ M=$(shell pwd) modules
clean:
	make -C /usr/src/linux-5.11.3/ M=$(shell pwd) clean
\end{lstlisting}

\begin{figure}[H]
    \centering
    \includegraphics[width=380pt]{jiff.png}
    \caption{Jiffies Module}
    \label{7223}
\end{figure}

\subsection{Task 3 - Seconds Module}


\begin{lstlisting}[language = c]
#include <linux/init.h>
#include <linux/module.h>
#include <linux/kernel.h>
#include <linux/proc_fs.h>
#include <asm/uaccess.h>
#include <linux/jiffies.h>
#include <asm/param.h>

#define BUFFER_SIZE 128

#define PROC_NAME "seconds"

unsigned long jiffies_load;

ssize_t proc_read(struct file *file, char *buf, size_t count, loff_t *pos);

static struct proc_ops proc_ops = {
        .proc_read = proc_read
};

int proc_init(void)
{
        proc_create(PROC_NAME, 0, NULL, &proc_ops);
        printk(KERN_INFO "/proc/%s created\n", PROC_NAME);
        jiffies_load = jiffies;
	return 0;
}

void proc_exit(void) {
        remove_proc_entry(PROC_NAME, NULL);
        printk( KERN_INFO "/proc/%s removed\n", PROC_NAME);
}

ssize_t proc_read(struct file *file, char __user *usr_buf, size_t count, loff_t *pos)
{
        int rv = 0;
        char buffer[BUFFER_SIZE];
        static int completed = 0;
        unsigned long seconds_since_load = (jiffies - jiffies_load )/ HZ;
        if (completed) {
                completed = 0;
                return 0;
        }
        completed = 1;
        rv = sprintf(buffer, "The number of seconds: %lu\n", seconds_since_load);
        copy_to_user(usr_buf, buffer, rv);
        return rv;
}

module_init( proc_init );
module_exit( proc_exit );

MODULE_LICENSE("GPL");
MODULE_DESCRIPTION("Seconds Module");
MODULE_AUTHOR("MisakaCenter");
\end{lstlisting}

\begin{lstlisting}
obj-m := seconds.o
all:
	make -C /usr/src/linux-5.11.3/ M=$(shell pwd) modules
clean:
	make -C /usr/src/linux-5.11.3/ M=$(shell pwd) clean
\end{lstlisting}

\begin{figure}[H]
    \centering
    \includegraphics[width=380pt]{sec.png}
    \caption{Seconds Module}
    \label{722ds3}
\end{figure}

\end{document}